% !TeX spellcheck = de_DE
%%%%%%%%%%%%%%%%%%%%%%%%%%%%%%%%%%%%%%%%%%%%%%%%%%%%%%%%%%%%%%%%%%%%%%%%%%%%%%%%%%%%%%%%%
%   _____                      _ _                        
%  / ____|                    | | |                       
% | |  __ _ __ _   _ _ __   __| | | __ _  __ _  ___ _ __  
% | | |_ | '__| | | | '_ \ / _` | |/ _` |/ _` |/ _ \ '_ \ 
% | |__| | |  | |_| | | | | (_| | | (_| | (_| |  __/ | | |
%  \_____|_|   \__,_|_| |_|\__,_|_|\__,_|\__, |\___|_| |_|
% 										  __/ |           
% 										 |___/            
%%%%%%%%%%%%%%%%%%%%%%%%%%%%%%%%%%%%%%%%%%%%%%%%%%%%%%%%%%%%%%%%%%%%%%%%%%%%%%%%%%%%%%%%%
%	Hier soll mathematisches, technisches, algorithmisches und anderes Wissen erklärt werden, aber nur soviel, wie für 
%	das Verständnis der Arbeit gebraucht wird. Als bekanntes Vorwissen kann der eigene Wissensstand vor Antritt der 
%	Arbeit angesehen werden. Weiterführendes Grundlagenwissen oder detaillierte mathematische Herleitungen in den 
%	Anhang verschieben. Die Arbeit ist kein Lehrbuch! Eventuell können hier Erklärungen der verwendeten Fachbegriffe 
%	stehen (oder am Anfang des MethodikKapitels).
%%%%%%%%%%%%%%%%%%%%%%%%%%%%%%%%%%%%%%%%%%%%%%%%%%%%%%%%%%%%%%%%%%%%%%%%%%%%%%%%%%%%%%%%%
\chapter{Theoretische Grundlagen} \label{3.theoretischeGrundlagen}

	\section{Placeholder}
		\lipsum