% !TeX spellcheck = de_DE
%%%%%%%%%%%%%%%%%%%%%%%%%%%%%%%%%%%%%%%%%%%%%%%%%%%%%%%%%%%%%%%%%%%%%%%%%%%%%%%%%%%%%%%%%
%										    _ _       
%										   | | |      
%	__   _____ _ ____      ____ _ _ __   __| | |_ ___ 
%	\ \ / / _ \ '__\ \ /\ / / _` | '_ \ / _` | __/ _ \
%	 \ V /  __/ |   \ V  V / (_| | | | | (_| | ||  __/
%	  \_/ \___|_|    \_/\_/ \__,_|_| |_|\__,_|\__\___|
%
%				   _          _ _             
%		/\        | |        (_) |            
%	   /  \   _ __| |__   ___ _| |_ ___ _ __  
%	  / /\ \ | '__| '_ \ / _ \ | __/ _ \ '_ \ 
%	 / ____ \| |  | |_) |  __/ | ||  __/ | | |
%	/_/    \_\_|  |_.__/ \___|_|\__\___|_| |_|
%%%%%%%%%%%%%%%%%%%%%%%%%%%%%%%%%%%%%%%%%%%%%%%%%%%%%%%%%%%%%%%%%%%%%%%%%%%%%%%%%%%%%%%%%
%	Es soll aufgezeigt werden, wer sich bereits mit dem Thema oder ähnlichen verwandten Themen auseinandergesetzt hat,
%	welche Lösungswege beschrieben wurden und was die Verbindung der jeweiligen Arbeit zur eigenen ist. Beachten Sie
%	die “4 W” als Gedankenstütze: 
%		Welches Problem wurde angegangen? 
%		Wie wurde das Problem gelöst? 
%		Was hat es gebracht? 
%		Wie steht es in Verbindung mit der eigenen Arbeit? 
%	In diesem Kapitel ist es besonders schwierig, einen roten Faden zu erzeugen und zu verhindern, dass der Text zu 
%	einer Paperauflistung verkommt. Es bieten sich Strategien an, die auch kombinierbar sind: Chronologisch oder 
%	aspektorientiert. Bei der chronologischen Auflistung beschreibt man verwandte Arbeiten zeitlich sortiert und gibt 
%	dadurch einen historischen Abriss über die Lösungsansätze des betrachteten Problems. Typischerweise beschreibt 
%	man die zeitlich erste Quelle genauer, sowie die zeitlich näher folgenden Quellen. Anschließend kann man etwas 
%	springen und sich auf Meilensteine konzentrieren. Schließlich sollte der aktuelle Stand wieder genauer betrachtet 
%	werden. Die zweite Strategie, aspektorientiertes Zitieren, sieht eine Unterteilung der Paper in Aspekte des 
%	eigenen Problems vor. Geht es zum Beispiel um Volume-Rendering mit globaler Beleuchtung, sollten Papers zum 
%	Volume-Rendering allgemein, dann Quellen zu erweiterten Methoden und schließlich Papers zur Integration von 
%	globaler Beleuchtung nacheinander (eventuell sogar in getrennten Abschnitten) vorgestellt werden.
%%%%%%%%%%%%%%%%%%%%%%%%%%%%%%%%%%%%%%%%%%%%%%%%%%%%%%%%%%%%%%%%%%%%%%%%%%%%%%%%%%%%%%%%%
\chapter{Verwandte Arbeiten} \label{2.verwandteArbeiten}

	\section{Placeholder}
		\lipsum