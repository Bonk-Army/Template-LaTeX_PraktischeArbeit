% !TeX spellcheck = de_DE
%%%%%%%%%%%%%%%%%%%%%%%%%%%%%%%%%%%%%%%%%%%%%%%%%%%%%%%%%%%%%%%%%%%%%%%%%%%%%%%%%%%%%%%%%		
%	 ______                _           _              
%	|  ____|              | |         (_)             
%	| |__   _ __ __ _  ___| |__  _ __  _ ___ ___  ___ 
%	|  __| | '__/ _` |/ _ \ '_ \| '_ \| / __/ __|/ _ \
%	| |____| | | (_| |  __/ |_) | | | | \__ \__ \  __/
%	|______|_|  \__, |\___|_.__/|_| |_|_|___/___/\___|
%				 __/ |                                
%				|___/                                 
%%%%%%%%%%%%%%%%%%%%%%%%%%%%%%%%%%%%%%%%%%%%%%%%%%%%%%%%%%%%%%%%%%%%%%%%%%%%%%%%%%%%%%%%%
%	Hier erfolgt eine objektive Darstellung der Ergebnisse. Dabei ist eine Unterteilung in quantitative Evaluation 
%	(Erzeugung von Messdaten) und qualitative Evaluation (Umfragen, Expertenfeedback, Beschreibung von Besonderheiten) 
%	sinnvoll. Pro evaluiertem Sachverhalt sollte zunächst das Ergebnis (z.B. als Abbildung oder Tabelle) gezeigt 
%	werden, danach wird es beschrieben und erst anschließend gedeutet. Eine Bewertung findet noch nicht statt.
%%%%%%%%%%%%%%%%%%%%%%%%%%%%%%%%%%%%%%%%%%%%%%%%%%%%%%%%%%%%%%%%%%%%%%%%%%%%%%%%%%%%%%%%%
\chapter{Ergebnisse} \label{6.ergebnisse}

\section{Placeholder}
\lipsum