% !TeX spellcheck = de_DE
%%%%%%%%%%%%%%%%%%%%%%%%%%%%%%%%%%%%%%%%%%%%%%%%%%%%%%%%%%%%%%%%%%%%%%%%%%%%%%%%%%%%%%%%% 
%	 ______        _ _   
%	|  ____|      (_) |  
%	| |__ __ _ _____| |_ 
%	|  __/ _` |_  / | __|
%	| | | (_| |/ /| | |_ 
%	|_|  \__,_/___|_|\__|
%%%%%%%%%%%%%%%%%%%%%%%%%%%%%%%%%%%%%%%%%%%%%%%%%%%%%%%%%%%%%%%%%%%%%%%%%%%%%%%%%%%%%%%%%
%	Die Arbeit wird kurz zusammengefasst und bewertet, die Ergebnisse/Lösungen in einem Fazit kondensiert (hierbei den 
%	Bogen zu den in der Einleitung aufgeworfenen Problemstellungen spannen). Es kann eine Einschätzung zur allgemeinen 
%	Nützlichkeit der entwickelten Verfahren gegeben werden. 
%		Tipp: Enden Sie mit etwas Positivem! Zeigen Sie im Fazit zunächst die „guten“ Dinge, danach die „schlechten“. 
%		Für letztere merken Sie an, dass sie lösbar sind und dass die entwickelten Verfahren nichtsdestotrotz 
%		vielversprechend sind.
%%%%%%%%%%%%%%%%%%%%%%%%%%%%%%%%%%%%%%%%%%%%%%%%%%%%%%%%%%%%%%%%%%%%%%%%%%%%%%%%%%%%%%%%%
\chapter{Zusammenfassung/Fazit} \label{8.zusammenfassung}

\section{Placeholder}
\lipsum