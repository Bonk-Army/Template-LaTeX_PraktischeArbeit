%%%%%%%%%%%%%%%%%%%%%%%%%%%%%%%%%%%%%%%%%%%%%%%%%%%%%%%%%%%%%%%%%%%%%%%%%%%%%%%
%% Descr:       Vorlage für Berichte der DHBW-Karlsruhe, Abstract
%% Author:      Prof. Dr. Jürgen Vollmer, vollmer@dhbw-karlsruhe.de
%% $Id: abstract.tex,v 1.11 2020/03/13 14:24:42 vollmer Exp $
%% -*- coding: utf-8 -*-
%%%%%%%%%%%%%%%%%%%%%%%%%%%%%%%%%%%%%%%%%%%%%%%%%%%%%%%%%%%%%%%%%%%%%%%%%%%%%%%

\vspace*{5cm}
%\vspace*{\fill}

\begingroup
	\centering
	\begin{abstract}
		\addcontentsline{toc}{chapter}{Zusammenfassung}
		Dieses \LaTeX-Dokument kann als Vorlage für einen Praxis- oder Projektbericht, eine Studien- oder Bachelorarbeit dienen.
		Zusammengestellt von Prof.\,Dr.\,Jürgen Vollmer \email{juergen.vollmer@dhbw-karlsruhe.de}
		\url{https://www.karlsruhe.dhbw.de} modifiziert von Nico Holzhäuser \email{nico.holzhaeuser@abilis.de} \url{https://abilis.de}. \\
		Die jeweils aktuellste Version dieses \LaTeX-Paketes ist immer
		auf der \emph{FAQ-Seite} des Studiengangs Informatik zu finden:
		\url{https://www.karlsruhe.dhbw.de/inf/studienverlauf-organisatorisches.html} $\to$ \emph{Formulare und Vorlagen}.
		\centering Stand \verb+$Date: 2020/09/17 15:07:45 $+
		\centering Stand DHBW \verb+$Date: 2020/03/13 15:07:45 $+
	\end{abstract}

	\vspace*{0.5cm}
	\selectlanguage{english} 
	
	\begin{abstract}
		This LATEX document can serve as a template for a practice or project report, a
		study or bachelor thesis. Compiled by Prof.\,Dr.\,Jürgen Vollmer \email{juergen.vollmer@dhbw-karlsruhe.de} \url{https://www.karlsruhe.dhbw.de} modified by
		Nico Holzhäuser \email{nico.holzhaeuser@abilis.de} \url{https://abilis.de}.
		The latest version of this \LaTeX-package can always be found on the FAQ page of the
		Computer Science program:
		\url{https://www.karlsruhe.dhbw.de/inf/studienverlauf-organisatorisches.html} $\to$ \emph{Formulare und Vorlagen}.
		\centering Stand \verb+$Date: 2020/09/17 15:07:45 $+
		\centering Stand DHBW \verb+$Date: 2020/03/13 15:07:45 $+
	\end{abstract}

	\selectlanguage{ngerman} 
	
	\vfill
	
	\begin{keywords}
		Hier, sollten, eigentlich, Keywords, stehen
	\end{keywords}
\endgroup

\vspace*{\fill}