% !TeX spellcheck = de_DE
%%%%%%%%%%%%%%%%%%%%%%%%%%%%%%%%%%%%%%%%%%%%%%%%%%%%%%%%%%%%%%%%%%%%%%%%%%%%%%%%%%%%%%%%%
%  ______ _       _      _ _                     
% |  ____(_)     | |    (_) |                    
% | |__   _ _ __ | | ___ _| |_ _   _ _ __   __ _ 
% |  __| | | '_ \| |/ _ \ | __| | | | '_ \ / _` |
% | |____| | | | | |  __/ | |_| |_| | | | | (_| |
% |______|_|_| |_|_|\___|_|\__|\__,_|_| |_|\__, |
%										    __/ |
%										   |___/ 
%%%%%%%%%%%%%%%%%%%%%%%%%%%%%%%%%%%%%%%%%%%%%%%%%%%%%%%%%%%%%%%%%%%%%%%%%%%%%%%%%%%%%%%%%
% 	Ganz am Anfang der schriftlichen Ausarbeitung soll die Einleitung eine griffige Motivation für die Aufgabe geben 
%	und elegant in das Thema einführen. Hier wird die Problemstellung im Kontext einer Anwendung dargestellt und der 
%	Inhalt der Arbeit kurz vorweggenommen. Welches Problem wurde gelöst, warum ist das relevant? Wie ist die 
%	Vorgehensweise? Was wurde thematisch eingegrenzt, was wurde ausgegrenzt? Wichtig ist es, den eigenen Beitrag in 
%	wenigen prägnanten Sätzen herauszuarbeiten. (Das Fazit soll sich zum Schluss auf diese Beiträge beziehen, um der %
%	Arbeit eine erzählerische Klammer zu geben.) Die Einleitung endet mit einem Überblick über die Arbeit – hierbei 
%	werden die Inhalte der einzelnen Kapitel knapp umschrieben (vermeiden Sie hier triviale Aussagen wie "Im 
%	Ergebniskapitel 7 werden die Ergebnisse präsentiert.") Tipp: Vermeiden Sie Standard-Intros wie „XY ist in der 
%	Computergraphik ein wichtiges Anwendungsfeld“ oder „In der Computergraphik ist XY nicht mehr wegzudenken“.
%%%%%%%%%%%%%%%%%%%%%%%%%%%%%%%%%%%%%%%%%%%%%%%%%%%%%%%%%%%%%%%%%%%%%%%%%%%%%%%%%%%%%%%%%
\chapter{Einleitung} \label{1.einleitung}
	
	\section{Placeholder}
		\lipsum \gls{real number} \index{TEST}