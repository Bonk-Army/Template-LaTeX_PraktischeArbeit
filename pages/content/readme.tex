%======================================================================
%	 ______                                                  
%	|  ____|                                                 
%	| |__   _ __ __ _  __ _  ___ _ __  _____   _ _ __   __ _ 
%	|  __| | '__/ _` |/ _` |/ _ \ '_ \|_  / | | | '_ \ / _` |
%	| |____| | | (_| | (_| |  __/ | | |/ /| |_| | | | | (_| |
%	|______|_|  \__, |\__,_|\___|_| |_/___|\__,_|_| |_|\__, |
%				 __/ |                                  __/ |
%				|___/                                  |___/ 
%
%----------------------------------------------------------------------
% Descripton : Ergänzende Dinge
%======================================================================

\chapter{Ergänzungen}
	\section{Nützliche Links}
	\begin{itemize}
		\item \url{http://www.network-science.de/ascii/} Um Ascii Header zu Erzeugen, verwendete Methode : \hk{big}
	\end{itemize}
	
	\section{Glossary Test}
	\gls{real number} \& \glspl{real number}
	
	\section{Todo Test}
	\todo[inline]{The original todo note withouth changed colours.\newline Here's another line.}
	\lipsum[11]\unsure{Is this correct?}
	\lipsum[11]\change{Change this!}
	\lipsum[11]\info{This can help me in chapter seven!}
	\lipsum[11]\improvement{This really needs to be improved!\newline What was I thinking?!}
	\lipsum[11]\thiswillnotshow{This is hidden since option `disable' is chosen!}
	\lipsum[11]\improvement[inline]{The following section needs to be rewritten!}
	
	\section{Test Online Quellen Zitat}
	Wichtig bei Onlinequellen immer eine Kopie, unter dem Ordner Test \hk{1\_Sources} speichern. Am besten als Websitedatei und nochmal als PDF.\cite{qs.koeln}