%======================================================================
%	 ______                                                  
%	|  ____|                                                 
%	| |__   _ __ __ _  __ _  ___ _ __  _____   _ _ __   __ _ 
%	|  __| | '__/ _` |/ _` |/ _ \ '_ \|_  / | | | '_ \ / _` |
%	| |____| | | (_| | (_| |  __/ | | |/ /| |_| | | | | (_| |
%	|______|_|  \__, |\__,_|\___|_| |_/___|\__,_|_| |_|\__, |
%				 __/ |                                  __/ |
%				|___/                                  |___/ 
%
%----------------------------------------------------------------------
% Descripton : Ergänzende Dinge
%======================================================================

\chapter{Ergänzungen}
	\epigraph{I recall seeing a package to make quotes}{Snowball}
	\section{Nützliche Links}
	\begin{itemize}
		\item \url{http://www.network-science.de/ascii/} Um Ascii Header zu Erzeugen, verwendete Methode : \hk{big}
	\end{itemize}
	
	\section{Glossary Test}
	\gls{real number} \& \glspl{real number}
	
	\section{Todo Test}
	\todo[inline]{The original todo note withouth changed colours.\newline Here's another line.}
	\lipsum[11]\unsure{Is this correct?}
	\lipsum[11]\change{Change this!}
	\lipsum[11]\info{This can help me in chapter seven!}
	\lipsum[11]\improvement{This really needs to be improved!\newline What was I thinking?!}
	\lipsum[11]\thiswillnotshow{This is hidden since option `disable' is chosen!}
	\lipsum[11]\improvement[inline]{The following section needs to be rewritten!}
	
	\section{Test der Theorem Umgebungen}
	Alle Umgebungen sind in der Datei \enquote{config/Style.sty} definiert. Und die Formatierung kann mit Hilfe des \href{Dokumentes}{https://www.math.uni-bielefeld.de/~rost/amslatex/doc/amsthdoc.pdf} angepasst werden.
	\begin{definition}
		\begin{equation}\label{equation}
			1 + 1 = 2,
		\end{equation}\eqdesc{Im Formelverzeichnis}
		and
		\begin{equation*}
			2 + 2 = 4.
		\end{equation*}
	\end{definition}
	
	\begin{beispiel}
		Das ist ein Beispiel mit dem Beweis in \cite{dante.2010a}
	\end{beispiel}
	
	\begin{bemerkung}
		Das ist eine Bemerkung
	\end{bemerkung}
	
	\section{Black out Test}
	Blackout eines kompletten Absatzes mit allen Makros etc. \\
	\begin{censorenv}
		\ac{WWW} \& \cite{dante.2010a} \footnote{hehehhehe}
	\end{censorenv}
	
	
	\xblackout{Lorem ipsum dolor sit amet, consetetur sadipscing elitr, sed diam nonumy eirmod tempor invidunt ut labore et dolore magna aliquyam erat, sed diam voluptua. At vero eos et accusam et justo duo dolores et ea rebum. Stet clita kasd gubergren, no sea takimata sanctus est Lorem ipsum dolor sit amet. Lorem ipsum dolor sit amet, consetetur sadipscing elitr, sed diam nonumy eirmod tempor invidunt ut labore et dolore magna aliquyam erat, sed diam voluptua. At vero eos et accusam et justo duo dolores et ea rebum.} \blackout{Stet clita kasd gubergren, no sea takimata sanctus est Lorem ipsum dolor sit amet. Lorem ipsum dolor sit amet, consetetur sadipscing elitr, sed diam nonumy eirmod tempor invidunt ut labore et dolore magna aliquyam erat, sed diam voluptua. At vero eos et accusam et justo duo dolores et ea rebum. Stet clita kasd gubergren, no sea takimata sanctus est Lorem ipsum dolor sit amet.}
	
	Duis autem vel eum iriure dolor in hendrerit in vulputate velit esse molestie consequat, vel illum dolore eu feugiat nulla facilisis at vero eros et accumsan et iusto odio dignissim qui blandit praesent luptatum zzril delenit augue duis dolore te feugait nulla facilisi. Lorem ipsum dolor sit amet, consectetuer adipiscing elit, sed diam nonummy nibh \censor{euismod tincidunt} ut laoreet dolore magna aliquam erat volutpat.
	
	Ut wisi enim ad minim veniam, quis nostrud exerci tation ullamcorper suscipit lobortis nisl ut aliquip ex ea commodo consequat. Duis autem vel eum iriure dolor in hendrerit in vulputate velit esse molestie consequat, vel illum dolore eu feugiat nulla facilisis at vero eros et accumsan et iusto odio dignissim qui blandit praesent luptatum zzril delenit augue duis dolore te feugait nulla facilisi.
	
	\section{Test Online Quellen Zitat}
	Wichtig bei Onlinequellen immer eine Kopie, unter dem Ordner Test \hk{1\_Sources} speichern. Am besten als Websitedatei und nochmal als PDF.\cite{qs.koeln}
	
	
	\section{Test große Anführungszeichen z.B. für die Einleitung}
	\begin{fancyquotes}
		\vspace{0.5cm} \\
		\label{sec:einleitungsbeispiel}\lipsum[50]\\
		\vspace{0.5cm}
	\end{fancyquotes}
	
	
	\section{Test Formatierung Forschungsfragen}
	
	% https://www.bachelorprint.de/wissenschaftliches-schreiben/forschungsfrage-formulieren/
	% https://www.utb.de/doi/book/10.36198/9783838554389
	% EDU Zugang bei UTB mit DHBW Login möglich!
	
	\forschungsfrage{Primäre}{Gestaltung}{Welche Maßnahmen/Strategien sollten Großunternehmen ergreifen, wenn sie auf dem chinesischen Markt erfolgreich bestehen wollen?}
	
	\vspace{1cm}
	
	\begin{UmgebungForschungsfrage}[Gestaltung]
		Welche Maßnahmen/Strategien sollten Großunternehmen ergreifen, wenn sie auf dem chinesischen Markt erfolgreich bestehen wollen?
	\end{UmgebungForschungsfrage}


	\section{Coole Formel mit Quelle}
	\begin{equation}\label{eq:arithStand}
		\sigma _{a} := \sqrt{\frac{1}{n}\sum^n_{j=1}(x_j-\bar x_{n})^2}
		\vadjust{\smallskip \hbox to \linewidth{\hfill\autocite[vgl. ][21]{henzeStochastikFuerEinsteiger.2021}}}
	\end{equation} \eqdesc{Berechnung der arithmetischen Standardabweichung}

	\section{Methodenspektrum}
	Vielen Dank an Krissi für das Methodenspektrum in TikZ.
	\begin{figure}[h]
		\centering
		\resizebox{0.65\linewidth}{!}{%
			\begin{tikzpicture}[remember picture, scale=1,font=\small\ttfamily,]
	%Koordinatensystem
	\draw (-7,7) -- (7,7) -- (7,-7) -- (-7,-7) -- (-7,7);
	\draw[dashed] (0,-7) -- (0,7);
	\draw[dashed] (-7,0) -- (7,0);
	%Labels links
	\node[anchor=east] (quantitativ) at (-7,3.5) {quantitativ};
	\node[rectangle, anchor=east, align=center] (degree of formalization) at (-7.5,0) {\textbf{Formalisierungs-}\\ \textbf{grad}};
	\node[anchor=east] (qualitativ) at (-7,-3.5) {qualitativ};
	%Labels unten
	\node[anchor=north] (behaviorist) at (-3.5,-7) {behavioristisch};
	\node[anchor=north] (degree of formalization) at (0,-7.5) {\textbf{Paradigma}};
	\node[anchor=north] (constructivist) at (3.5,-7) {konstruktivistisch};
	\node[circle, fill=gray!30, minimum size=0.7cm] (a) at (-5,3) {};
	\node[anchor=south] (a) at (-5,3) {Quantitative};
	\node (a) at (-5,3) {Querschnitts-};
	\node[anchor=north] (a) at (-5,3) {analyse};
	\node[circle, fill=gray!30, minimum size=0.7cm] (a) at (-5,-3) {};
	\node[anchor=south] (a) at (-5,-3) {Qualitative};
	\node (a) at (-5,-3) {Querschnitts-};
	\node[anchor=north] (a) at (-5,-3) {analyse};
	\node[circle, fill=gray!30, minimum size=0.7cm] (a) at (-1,-3) {};
	\draw[-{Triangle[width=10pt,length=8pt]}, line width=6pt, color=gray!30](-1,-3) -- (0, -3);
	\draw[-{Triangle[width=10pt,length=8pt]}, line width=6pt, color=gray!30](-1,-3) -- (-2, -3);
	%\node[anchor=south] (a) at (-1,-3.1) {case};
	\node[anchor=north] (a) at (-1,-2.75) {Fallstudie};
	\node[circle, fill=gray!30, minimum size=0.7cm] (a) at (0,-5) {};
	\node (a) at (0,-5) {Ethnographie};
	\node[circle, fill=gray!30, minimum size=0.7cm] (a) at (-5,-1) {};
	\node[anchor=south] (a) at (-5,-1.1) {Grounded};
	\node[anchor=north] (a) at (-5,-0.9) {Theory};
	\node[circle, fill=gray!30, minimum size=0.7cm] (a) at (-4,4) {};
	\node[anchor=south] (a) at (-4,4.1) {Feld};
	\node[anchor=north] (a) at (-4,4.3) {experiment};
	\node[circle, fill=gray!30, minimum size=0.7cm] (a) at (-2,5) {};
	\draw[-{Triangle[width=10pt,length=8pt]}, line width=6pt, color=gray!30](-2,5) -- (-1,5);
	\draw[-{Triangle[width=10pt,length=8pt]}, line width=6pt, color=gray!30](-2,5) -- (-3,5);
	\node[anchor=south] (a) at (-2,4.9) {Labor-};
	\node[anchor=north] (a) at (-2,5.1) {experimente};
	\node[circle, fill=gray!30, minimum size=0.7cm] (a) at (5,5) {};
	\node[anchor=south] (a) at (5,5) {Formal-};
	\node (a) at (5,5) {deduktive};
	\node[anchor=north] (a) at (5,5) {Analyse};
	\node[circle, fill=gray!30, minimum size=0.7cm] (a) at (5,0) {};
	\node[anchor=south] (a) at (5,0) {Konzeptionell-};
	\node (a) at (5,0) {deduktive};
	\node[anchor=north] (a) at (5,0) {Analyse};
	\node[circle, fill=gray!30, minimum size=0.7cm] (a) at (5,-5) {};
	\node[anchor=south] (a) at (5,-5) {Argumentative-};
	\node (a) at (5,-5) {deduktive};
	\node[anchor=north] (a) at (5,-5) {Analyse};
	\node[circle, fill=gray!30, minimum size=0.7cm] (a) at (2,1) {};
	\draw[-{Triangle[width=10pt,length=8pt]}, line width=6pt, color=gray!30](2,1) -- (1,1);
	\draw[-{Triangle[width=10pt,length=8pt]}, line width=6pt, color=gray!30](2,1) -- (3,1);
	\node[anchor=south] (a) at (2,0.9) {Referenz-};
	\node[anchor=north] (a) at (2,1.1) {modellierung};
	\node[circle, fill=gray!30, minimum size=0.7cm] (a) at (5,4) {};
	\node (a) at (5,4) {Simulation};
	\node[circle, fill=blue!30, minimum size=0.7cm] (a) at (3,-2) {};
	\draw[-{Triangle[width=10pt,length=8pt]}, line width=6pt, color=blue!30](3,-2) -- (2,-2);
	\draw[-{Triangle[width=10pt,length=8pt]}, line width=6pt, color=blue!30](3,-2) -- (4,-2);
	\node (a) at (3,-2) {Prototyping};
	\node[circle, fill=gray!30, minimum size=0.7cm] (a) at (3,-3) {};
	\draw[-{Triangle[width=10pt,length=8pt]}, line width=6pt, color=gray!30](3,-3) -- (2,-3);
	\draw[-{Triangle[width=10pt,length=8pt]}, line width=6pt, color=gray!30](3,-3) -- (4,-3);
	\node[anchor=south] (a) at (3,-3.1) {Aktions-};
	\node[anchor=north] (a) at (3,-2.9) {forschung};
\end{tikzpicture}   
		}
		\caption{\label{fig:Methodenspektrum} Methodenspektrum der (Wirtschafts-)Informatik \autocite[vgl.][14]{wildeMethodenspektrumWirtschaftsinformatikUeberblick.2006}}
	\end{figure}